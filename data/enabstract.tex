% !Mode:: "TeX:UTF-8"

%%% 此处是英文摘要的格式设置,数据部分在下面 %%%
\makeatletter

\long\def\enabstract#1{\long\def\@enabstract{#1}}\long\def\@enabstract{}
\def\enkeywords#1{\def\@enkeywords{#1}}\def\@enkeywords{}
\def\enpageheader#1{\def\@enpageheader{#1}}\def\@enpageheader{}

% 定义makeenabstract以生成英文摘要
\def\makeenabstract{
	
	\clearpage
	
	% 使用fancy来设置页眉页脚,自定义了新的格式——englishAbstract
	\pagestyle{fancy}
	\fancypagestyle{englishAbstract}{
		\fancyhead[C]{\wuhao \@enpageheader}		% 使用英文论文名称作为页眉
		\fancyfoot[C]{\song\wuhao ~\thepage~}
	}
	
	\addcontentsline{toc}{chapter}{Abstract}
	\chapter*{\centering\erhao \bf{Abstract}}		% 标题,二号
	\vspace{\baselineskip} 							% 空两行
	\vspace{\baselineskip}
	
	\@enabstract									% 英文摘要正文
	\thispagestyle{englishAbstract}					% 应用englishAbstract
	\vspace{\baselineskip}
	
	%\hangafter=1\hangindent=60pt\noindent  %如果取消该行注释,KEY WORDS换行时将会自动缩进
	\noindent
	{\xiaosi\textbf{Key Words:} \@enkeywords}		% 英文关键字
	
}
\makeatother


%%% 英文摘要数据部分 %%%

\enabstract{英文摘要写在此处}
\enpageheader{英文页眉标题}
\enkeywords{keyworkds1;~~keyworkds2;~~keyworkds3}

% 使用makeenabstract生成英文摘要
\makeenabstract